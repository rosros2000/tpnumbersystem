% !TeX encoding = UTF-8
% !TeX TS-program = xelatex
% !TeX root = numbersystem.tex


\documentclass{article} 


\usepackage{textcomp}
\usepackage{graphicx} 
\usepackage{fontspec} 
\usepackage{multicol} 
\usepackage{hanging} 
\usepackage{titlesec} 
\usepackage{hyperref} 
\usepackage{fix-cm} 
\usepackage{xcolor}
\usepackage{setspace}
%\usepackage[italian]{babel}
\usepackage[left=1.0in,top=-0.2in,bottom=0.9in]{geometry} \titlespacing*{\section}{0pt}{0.3cm}{0.0cm} 
\usepackage{xeCJK}
\setCJKmainfont{AR PL UMing TW}

%\newfontface\tpf{linjapona49}

\newfontface\tpf{linja sike 5} 

\newcommand*\sitp[1]{\large{\tpf{#1}}}

\newcommand*\tpGlyph[2]{ 
	\begin{tabular}{c}\huge{\tpf #1} \\ \small #2 
		\vspace{2mm}
\end{tabular}} 

\renewcommand{\arraystretch}{1.2}

\makeatletter

\newcommand\HHUGE{\@setfontsize\Huge{50}{60}} \makeatother 

\title{Power-based number system \\{\HHUGE{\tpf{nasin nanpa sewi1}}} \\ V 0.27 $\alpha$}
\author{\textit{jan loje, Shaevor, jan Tamalu}} 



\begin{document} 
	
	
\maketitle

\setstretch{1.2}	

\section{Power-based number system for toki pona}
	
\emph{Written by jan loje with the help of jan Tamalu and Shaevor}
\emph{(mistakes are mine only)}

This system should be:

1. based on power-of-10 notation, a.k.a. scientific notation, 

2. easy to understand, learn, and use,

3. suitable for \textit{toki pona}.

NOTES: < >: \textit{read as}

\section{Vocabulary}
	
%The ideas for the choice of the numerals are indicated between square brackets {[} {]}

\vspace{10pt}
	
	\begin{tabular}{|c|c|c|l|} 
		\hline
		1 & wan & \sitp{wan} & one\tabularnewline
		\hline
		2 & tu & \sitp{tu} & two\tabularnewline
		\hline
		3 & sin & \sitp{sin} & three {[}3 lines{]}\tabularnewline
		\hline
		4 & lipu & \sitp{lipu} & four {[}4 sides{]}\tabularnewline
		\hline
		5 & luka & \sitp{luka} & five {[}toki pona hand{]}\tabularnewline
		\hline
		6 & pipi & \sitp{pipi} & six {[}6 elements{]}\tabularnewline
		\hline
		7 & len & \sitp{len} & seven {[}4 sides + 3 lines{]}\tabularnewline
		\hline
		8 & musi & \sitp{musi} & eight [two circles look like a kind of
		8]\tabularnewline
		\hline
		9 & suli & \sitp{suli} & nine {[}the "big" digit{]}\tabularnewline
		\hline
		10 & sewi & \sitp{sewi1} & 10 (base) followed by integer powers (1 is
		implicit): 2, 3, 4,... {[}raise{]}\tabularnewline
		\hline
		20 & tu sewi & \sitp{tu sewi1} & two × ten\tabularnewline
		\hline
		30 & sin sewi & \sitp{sin sewi1} & three × ten\tabularnewline
		\hline
		100 & sewi tu & \sitp{sewi1 tu} & 10\textsuperscript{2}\tabularnewline
		\hline
		300 & sin sewi tu & \sitp{sin sewi1 tu} & three × ten\textsuperscript{2}\tabularnewline
		\hline
		1000 & sewi sin & \sitp{sewi1 sin} & 10\textsuperscript{3}\tabularnewline
		\hline
		+ & en & \sitp{en} & addition\tabularnewline
		\hline
		- & weka & \sitp{weka} & negative {[}toki pona subtract{]}\tabularnewline
		\hline
		. & sike & \sitp{sike} & separator for decimal part\tabularnewline
		\hline
		№ & nanpa & \sitp{nanpa} & number prefix (ordinal)*\tabularnewline
		\hline
		\# & mute & \sitp{mute} & number prefix (cardinal)\tabularnewline
		\hline
	\end{tabular}

\vspace{5pt}	

	*NOTE: compare Philipino ika- or pang-, Malay and Indonesian ke-, Chinese 第

\newpage

\newgeometry{left=1.0in,top=1in,bottom=0.9in}



\section{Rationale}

\vspace{5pt}

This system might be a way to \textit{read} numbers and dates written with the digits (0-9) in \textit{toki pona} text. Additional meanings could be added to some already existing \textit{toki pona} words.


\section{Use}
	
\subsection{Prefixes (when needed)}

    \textbf{Ordinal and cardinal numbers}

	\vspace{3 pt}
	
	\textit{nanpa} {\sitp{nanpa}}:  ordinal number
	
	\textit{mute} {\sitp{mute}}: cardinal number
	
	\vspace{5 pt}
	
	{\sitp{ona li nanpa 5}}~<\textit{ona li nanpa luka}> it's the 5th (ordinal)
	
	{\sitp{ona li mute 5}}~<\textit{ona li mute luka}> it's 5 (cardinal) 
	
	
\subsection{Positional digits}
	
	\textbf{The values of digits are \emph{positional} (common usage)}

	\vspace{5 pt}
		
    That is 212 = 2 × 10\textsuperscript{2} + 1 × 10\textsuperscript{1} + 2 × 10\textsuperscript{0}
	
	\vspace{5 pt}
	
	12 <\textit{wan tu}>
	
	2024 <\textit{tu ala tu lipu}>
	
\subsection{Numbers as powers of 10}
    
   \textbf{ \textit{sewi} is the base 10 for all powers.}
    
    \vspace{5 pt}
	
	1000 = 10\textsuperscript{3} <\emph{sewi sin}>
	
	10 000 = 10\textsuperscript{4} <\emph{sewi lipu}>
	
	...
	
	1 000 000 000 = 10\textsuperscript{9}= \emph{sewi suli}
	
\vspace {12 pt}
	
	{\sitp{jan li jo e \$1,000,000,000}}~<\textit{jan li jo e mani Mewika pi mute sewi suli}>
	
\subsection{Very large (or small) numbers}

   \textbf{ Very large (or small) numbers can be expressed easily.}
    
    \vspace {6 pt}
	
	a googol = 10\textsuperscript{100} \textit{<sewi wan ala ala>} \textit{or} 10\textsuperscript{10\textsuperscript{2}} <\textit{sewi sewi tu}>
	
	\subsection{Composed numbers}
	
	\textbf{Numbers with multiplicative and additive values.}
	
	\vspace {6 pt}
	
	The number to the left of \emph{sewi} has multiplicative value.
	
		\vspace{5pt}
	
	The additive value of a number (sequence) is stated explicitly with
	\emph{en}.
     \vspace {6 pt}
   	
	4 000 000 012 = 4 × 10\textsuperscript{9}+ 12  <\emph{lipu sewi suli en	wan tu}>
	
	\subsection{Numbers with fractional parts}
	
	\textbf{Number with a fractional part separated by a decimal point.}
	
		\vspace{5pt}
		
	3.14 <\emph{sin sike wan lipu}>
	
		\vspace{5pt}
		
	3.14 = 314 × 10\textsuperscript{-2} <\emph{sin wan lipu sewi weka tu}>
	
	\subsection{Numbers with negative exponents}
	
	\textbf{Negative exponents are prefixed by \textit{weka}.}

		\vspace{5pt}
	
	6.62 × 10\textsuperscript{-34} <\emph{pipi sike pipi tu sewi weka sin lipu}>
	
	\subsection{Dates}
	
	\textbf{ISO 8601 system}
	\vspace{5pt}
	
	{\sitp{2024-05-12}}~<\emph{tenpo sike tu ala tu lipu \textbf{en} tenpo mun luka \textbf{en} tenpo suno wan tu}>
	
	\vspace{5pt}
	
	{\sitp{5-12 la ona li kama lon}}~<\emph{tenpo mun luka en tenpo suno wan tu la ona li kama lon ale}> His birthday is May 12th
	
	\vspace{5pt}
	
\end{document} 