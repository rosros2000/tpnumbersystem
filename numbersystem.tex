% !TeX encoding = UTF-8
% !TeX TS-program = xelatex
% !TeX root = numbersystem.tex
% 20240310 Sunday
% A chart of all the glyphs in Toki Pona, arranged by shape.
% Process this with 'lualatex'.
% aquatiki fork - expanded

\documentclass{article} 
\usepackage{textcomp}
\usepackage{graphicx} 
\usepackage{fontspec} 
\usepackage{multicol} 
\usepackage{hanging} 
\usepackage{titlesec} 
\usepackage{hyperref} 
\usepackage{fix-cm} 
\usepackage{xcolor}
\usepackage{setspace}
%\usepackage[italian]{babel}
\usepackage[left=1.0in,top=0in,bottom=0.6in]{geometry} \titlespacing*{\section}{0pt}{0.3cm}{0.0cm} 

%\newfontface\tpf{linjapona49}

\newfontface\tpf{linja sike 5} 

\newcommand*\tpGlyph[2]{ 
	\begin{tabular}{c}\huge{\tpf #1} \\ \small #2 
		\vspace{2mm}
\end{tabular}} 

\renewcommand{\arraystretch}{1.2}

\makeatletter

\newcommand\HHUGE{\@setfontsize\Huge{50}{60}} \makeatother 

\title{Power-based number system for toki pona \\{\HHUGE{\tpf{ nasin nanpa sewi1 pi toki-pona}}} \\ V 0.15 $\alpha$}
\author{\textit{jan loje, jan Tamalu}} 

\begin{document} 
	
	
\maketitle

\setstretch{1.2}	

\section{Toki pona power-based number system}
	
\emph{With the help of jan Tamalu}
\emph{(mistakes are mine only)}

This system should be:

1. based on the common decimal system and digits,

2. unambiguous,

3. easy to understand, learn, and use for all common non-scientific and non-mathematical purposes,

4. suitable for \textit{toki pona}.

\section{Vocabulary}
	
The ideas for the choice of the names are indicated between square 	brackets {[} {]}

\vspace{10pt}
	
	\begin{tabular}{|c|c|c|l|} 
		\hline
		1 & wan & \tpf{wan} & one\tabularnewline
		\hline
		2 & tu & \tpf{tu} & two\tabularnewline
		\hline
		3 & sin & \tpf{sin} & three {[}3 lines{]}\tabularnewline
		\hline
		4 & lipu & \tpf{lipu} & four {[}4 sides{]}\tabularnewline
		\hline
		5 & luka & \tpf{luka} & five\tabularnewline
		\hline
		6 & pipi & \tpf{pipi} & six {[}6 elements{]}\tabularnewline
		\hline
		7 & len & \tpf{len} & seven {[}4 sides + 3 lines{]}\tabularnewline
		\hline
		8 & musi & \tpf{musi} & eight; two circles look a kind of
		8\tabularnewline
		\hline
		9 & suli & \tpf{suli} & nine {[}the "big" digit{]}\tabularnewline
		\hline
		10 & sewi & \tpf{sewi1} & 10 (base) followed by integer powers (1 is
		implicit): 2, 3, 4,... {[}raise{]}\tabularnewline
		\hline
		20 & tu sewi & \tpf{tu sewi1} & two × ten\tabularnewline
		\hline
		30 & sin sewi & \tpf{sin sewi1} & three × ten\tabularnewline
		\hline
		100 & sewi tu & \tpf{sewi1 tu} & 10\textsuperscript{2}\tabularnewline
		\hline
		300 & sin sewi tu & \tpf{sin sewi1 tu} & three × ten\textsuperscript{2}\tabularnewline
		\hline
		1000 & sewi sin & \tpf{sewi1 sin} & 10\textsuperscript{3}\tabularnewline
		\hline
		+ & en & \tpf{en} & addition\tabularnewline
		\hline
		- & weka & \tpf{weka} & negative {[}subtract{]}\tabularnewline
		\hline
		. & sike & \tpf{sike} & separator for decimal part\tabularnewline
		\hline
		№ & nanpa & \tpf{nanpa} & number prefix (ordinal)*\tabularnewline
		\hline
		\# & mute & \tpf{mute} & number prefix (cardinal)\tabularnewline
		\hline
	\end{tabular}

\vspace{5pt}	

	*NOTE: compare Philipino ika- or pang-, Malay and Indonesian ke-

\newpage

\newgeometry{left=1.0in,top=1in,bottom=0.6in}


\section{Use}
	
\subsection{Prefixes (when needed)}

    \textbf{Ordinal and cardinal numbers}.
	
	\textit{nanpa} {\tpf{nanpa}}:  ordinal number
	
	\textit{mute} {\tpf{mute}}: cardinal number
	
	\vspace{5 pt}
	
	Ex. 1:
	
	\textit{ona li nanpa luka} = it's the 5th (ordinal)
	
	\textit{ona li mute luka} = it's 5 (cardinal) 
	
	
\subsection{Non-additive numbers}
	
	\textbf{Numbers are \emph{non-additive}}.
	
	\vspace{5 pt}
	
	120 = \emph{wan tu ala}
	
	2024 = \emph{tu ala tu lipu}
	
\subsection{Numbers as powers of 10}
    
   \textbf{ \textit{sewi} is the base 10 for all powers.}
    
    \vspace{5 pt}
	
	1000 = 10\textsuperscript{3}= \emph{sewi sin}
	
	10 000 = 10\textsuperscript{4}= \emph{sewi lipu}
	
	...
	
	1 000 000 000 = 10\textsuperscript{9}= \emph{sewi suli}
	
\vspace {12 pt}
	
	Ex. 2: \emph{jan li jo e \$1,000,000,000}
	
	\emph{jan li jo e mani Mewika pi mute sewi suli}
	
\subsection{Very large (or small) numbers}

   \textbf{ Very large (or small) numbers can be expressed easily.}
    
    \vspace {6 pt}
	
	a googol = 10\textsuperscript{100} = sewi wan ala ala 
	
	or 
	
	10\textsuperscript{10\textsuperscript{2}} = sewi sewi tu
	
	\subsection{Composed numbers}
	
	\textbf{Numbers with multiplicative and additive values.}
	
	\vspace {6 pt}
	
	The number to the left of \emph{sewi} has multiplicative value.
	
	The additive value of a number (sequence) is stated explicitly with
	\emph{en}.
   \vspace {6 pt}
   	
	4 000 000 012 = 4 × 10\textsuperscript{9}+ 12 = \emph{lipu sewi suli en
		wan tu}
	
	\subsection{Numbers with fractional parts}
	
	\textbf{Number with a fractional part separated by a decimal point.}
	
	3.14 = \emph{sin sike wan lipu}
	
	3.14 = 314 × 10\textsuperscript{-2} = \emph{sin wan lipu sewi weka tu}
	
	\subsection{Numbers with negative exponents}
	
	\textbf{Negative exponents are prefixed by \textit{weka}.}
	
	6.62 × 10\textsuperscript{-34} = \emph{pipi sike pipi tu sewi weka sin
		lipu}
	
	\subsection{Dates}
	
	\textbf{ISO 8601 system}
	\vspace{5pt}
	
	2024-05-12 = \emph{tenpo sike tu ala tu lipu en tenpo mun luka en tenpo
		suno wan tu}
	
	Ex. 3: \emph{05-12 ona li kama lon} = His birthday is May 12th
	
	\emph{tenpo mun luka en tenpo suno wan tu la ona li kama lon}

	\huge{\tpf{tenpo mun luka en tenpo suno wan tu la ona li kama lon}}
	
\end{document} 