% !TeX encoding = UTF-8
% !TeX TS-program = xelatex
% !TeX root = numbersystem-glossary.tex
% 20240304 Monday
% A chart of all the glyphs in Toki Pona, arranged by shape.
% Process this with 'lualatex'.
% aquatiki fork - expanded
\documentclass{article}
\usepackage{graphicx}
\usepackage{fontspec}
\usepackage{multicol}
\usepackage{hanging}
\usepackage{titlesec}
\usepackage{hyperref}
\usepackage{fix-cm}    
\usepackage{amssymb}
\usepackage{titlesec}
\usepackage[left=1.0in,top=0in,bottom=0.8in]{geometry}
\titlespacing*{\section}{0pt}{0.3cm}{0.0cm}
%\newfontface\tpf{linjapona49}
\newfontface\tpf{linja sike 5}
\newcommand*\tpGlyph[2]{
	\begin{tabular}{c}\Huge{\tpf #1} \\ \small #2 \vspace{2mm}\end{tabular}}

\makeatletter
\newcommand\HHUGE{\@setfontsize\Huge{40}{50}}
\makeatother  

\pagestyle{myheadings}
\markboth{}{\Huge{\tpf lipu-nimi pi nasin nanpa sewi1}}

\title{Power-based number system glossary \\[1ex]{\HHUGE{\tpf lipu-nimi pi nasin nanpa sewi1}} \\ V 0.15 $\alpha$} 
\author{\textit{jan loje}}

\begin{document}
\maketitle

\newcommand\tpDef[4]{\textbf{#1} {\Large\tpf #2} <\textsf{#3}> #4}

\newcommand\tpDefB[3]{\textbf{#1} {\Large\tpf #2} \textit{#3}}

\setlength{\parskip}{0.3em}

cf. = compare

|| = example(s)

< > = \textit{reads as}



\begin{multicols}{2}
\raggedright
\begin{hangparas}{1cm}{1}

\section*{NUMBERS AND SYMBOLS}

\tpDef {0}{ala}{ala}{zero}

\tpDef {1st}{nanpa wan}{nanpa wan}{first}

\tpDef {1}{wan}{wan}{one}

\tpDef {2}{tu}{tu}{two}

\tpDef {3}{sin}{sin}{three}

\tpDef {4}{lipu}{lipu}{four}

\tpDef {5}{luka}{luka}{five}

\tpDef {6}{pipi}{pipi}{six}

\tpDef {7}{len}{len}{seven}

\tpDef {8}{musi}{musi}{eight}

\tpDef {9}{suli}{suli}{nine}

\tpDef {10}{sewi1}{sewi}{ten, 10 base for powers || 10\textsuperscript{2} = 100 \textsf{sewi tu};  10\textsuperscript{6}= 1 million \textsf{<sewi pipi>}; a googol = 10\textsuperscript{10\textsuperscript{2}} \textsf{<sewi sewi tu>}}

\tpDef {add}{en}{en}{(additive use of numbers || 1003:\textit{ sewi sin en sin}}

\tpDef {negative}{weka}{weka}{-, sign of negative number; ||  6.62 x 10\textsuperscript{-34} \textsf{<pipi sike pipi tu sewa weka sin lipu>} }

\tpDef {cardinal}{mute}{mute}{(cardinal use of numbers) || \textit{ona li mute luka} = it's 5}

\tpDef {ordinal}{nanpa}{nanpa}{(ordinal use of numbers) || \textit{ona li nanpa luka} = it's the 5th}

\tpDef {hundred}{100}{sewi tu}{10\textsuperscript{2}}

\tpDef {million}{1,000,000}{sewi pipi}{10\textsuperscript{6}}

\tpDef {crore}{10,000,000}{sewi len}{10\textsuperscript{7}}

\tpDef {billion}{1,000,000,000}{sewi suli}{10\textsuperscript{9} || jan~li~jo~\$1,000,000,000  \textsf{<jan li jo e mani Mewika pi mute sewi suli>} he owns  \$1,000,000,000 }

\tpDef {googol}{10\textsuperscript{10\textsuperscript{2}}}{sewi sewi tu}{10\textsuperscript{100}}

\section*{DATES AND TIMES}
\vspace{5mm}

\subsection*{YEARS}

\vspace{5mm}

\tpDef {year}{tenpo sike, sike suno}{tenpo sike, sike suno}{time of one complete revolution of the Earth around the Sun}

\tpDef {century}{tenpo sike 100}{tenpo sike sewi tu, tenpo sike wan ala ala}{time of 100 complete revolutions of the Earth around the Sun}

\tpDef {millenium}{tenpo sike 1000}{tenpo sike sewi sin}{time of 1000 complete revolutions of the Earth around the Sun}

\tpDef {megayear}{tenpo sike 1,000,000}{tenpo sike sewi pipi}{time of 1,000,000 complete revolutions of the Earth around the Sun}

\tpDef {eon}{tenpo sike 1,000,000,000}{tenpo sike sewi suli}{time of 1,000,000,000 complete revolutions of the Earth around the Sun}

\tpDef {2024-05-12}{2024-05-12}{tenpo sike tu ala tu lipu en tenpo mun luka en tenpo suno wan tu}{May 12th, 2024 [ISO 8601]}

\end{hangparas}
\end{multicols}

\newpage
\newgeometry{left=0.6in,right=0.25in,top=1in,bottom=0.55in}

\begin{multicols}{2}
	\raggedright
	\begin{hangparas}{1cm}{1}

\subsection*{MONTHS}

\tpDef {month}{tenpo mun}{tenpo mun}{one of the 12 subdivisions of the year}

\tpDef {quarter}{tenpo mun mute sin}{tenpo mun mute sin}{one of the subdivisions of the year in 4 parts (3 months) cf. \textit{tenpo mun sin}: March}

\tpDef {first quarter}{tenpo mun mute sin nanpa wan}{tenpo mun mute sin nanpa wan}{January-February-March}

\tpDef {January}{tenpo mun wan}{tenpo mun wan}{1st month of the year}

\tpDef {February}{tenpo mun tu}{tenpo mun tu}{2nd month of the year}

\tpDef {March}{tenpo mun sin}{tenpo mun sin}{3rd month of the year}

\tpDef {April}{tenpo mun lipu}{tenpo mun lipu}{4th month of the year}

\tpDef {May}{tenpo mun luka}{tenpo mun luka}{5th month of the year}

\tpDef {June}{tenpo mun pipi}{tenpo mun pipi}{6th month of the year}

\tpDef {July}{tenpo mun len}{tenpo mun len}{7th month of the year}

\tpDef {August}{tenpo mun musi}{tenpo mun musi}{8th month of the year}

\tpDef {September}{tenpo mun suli}{tenpo mun suli}{9th month of the year}

\tpDef {October}{tenpo mun sewi1}{tenpo mun sewi}{10th month of the year}

\tpDef {November}{tenpo mun wan wan}{tenpo mun wan wan}{11th month of the year}

\tpDef {December}{tenpo mun wan tu}{tenpo mun wan tu}{12th month of the year}

\subsection*{WEEK AND WEEKDAYS}

\tpDef {week}{tenpo esun, sike esun}{tenpo esun, sike esun}{seven consecutive solar days }

\tpDef {Monday}{tenpo suno mun}{tenpo suno mun}{the first [ISO-8601] day of the week }

\tpDef {Tuesday}{tenpo suno seli}{tenpo suno seli}{the second [ISO-8601] day of the week }

\tpDef {Wednesday}{tenpo suno telo }{tenpo suno telo }{the third [ISO-8601] day of the week }

\tpDef {Thursday}{tenpo suno kasi}{tenpo suno kasi}{the fourth [ISO-8601] day of the week }

\tpDef {Friday}{tenpo suno kiwen}{tenpo suno kiwen}{the fifth [ISO-8601] day of the week }

\tpDef {Saturday}{tenpo suno ma}{tenpo suno ma}{the sixth [ISO-8601] day of the week }

\tpDef {Sunday}{tenpo suno esun}{tenpo suno esun}{the seventh [ISO-8601] day of the week}

\subsection*{DAYS}

\tpDef {day}{tenpo suno}{tenpo suno}{solar day, 24 hours | ALT time from sunrise to sunset (according to context)}

\tpDef {night}{tenpo pimeja, tenpo lape}{tenpo pimeja, tenpo lape}{time from sunset to sunrise}

\tpDef {sunset}{pini suno, weka suno}{pini suno, weka suno}{time of day when the sun disappears below the western horizon}

\tpDef {sunrise}{kama suno, open suno, sewi1 suno}{kama suno, open suno, sewi suno}{time of day when the sun appears above the eastern horizon}

\tpDef {today}{tenpo suno lon, tenpo suno ni}{tenpo suno lon, tenpo suno ni}{current solar day}

\tpDef {tomorrow}{tenpo suno kama}{tenpo suno kama}{next solar day after the current day}

\tpDef {yesterday}{tenpo suno pini}{tenpo suno pini}{previous solar day before the current day}

\tpDef {day after tomorrow}{tenpo suno pi kama tu}{tenpo suno pi kama tu}{second next solar day after the current day}

\tpDef {day before yesterday}{tenpo suno pi pini tu}{tenpo suno pi pini tu}{second previous solar day before the current day}

\tpDef {new year's (day)}{tenpo suno wan pi tenpo sike}{tenpo suno wan pi tenpo sike}{new year's, the first day of the year}


\subsection*{HOURS}

\tpDef {hour}{tenpo pi palisa lili}{tempo pi palisa lili}{1/24 of a day}

\tpDef {minute}{tenpo pi palisa suli}{tempo pi palisa suli}{1/60 of an hour}

\tpDef {second}{tenpo pi palisa tawa}{tempo pi palisa tawa}{1/60 of a minute}

\tpDef {noon}{tenpo suno wawa}{tenpo suno wawa}{12:00 || \textit{tempo ni li suno wawa}: it's 12 o'clock}

\tpDef {midnight}{tenpo pimeja wawa}{tenpo pimeja wawa}{24:00}

\tpDef {15:20}{15:20} {20 minutes past 3 o'clock P.M.}{3:20 P.M. ||\textit{ o mi kama lon poka lon tenpo wan luka en tu ale}: let's meet at 15:20}

\subsection*{OTHERS}

\tpDef {summer}{tenpo seli}{tenpo seli}{the warmest season of the year}

\tpDef {winter}{tenpo lete}{tenpo lete}{the coldest season of the year}

\tpDef {spring}{tenpo pi seli lili}{tenpo pi seli lili}{the season of the year between winter and summer}

\tpDef {autumn}{tenpo loje, tenpo loje jelo, tenpo pi lete lili}{tenpo loje, tenpo loje jelo, tenpo pi lete lili}{the season of the year between summer and winter}

\tpDef {now}{tenpo ni}{tempo ni}{at the present time or moment}

\tpDef {before}{tenpo pini, tenpo monsi}{tenpo pini, tenpo monsi}{before the present time or moment}

\tpDef {after}{tenpo kama}{tempo kama}{in a future time or moment}

\tpDef {soon}{tenpo kama lili}{tempo kama lili}{within a short time, or quickly}

\tpDef {never}{tenpo ala}{tempo ala}{at no time, in no way}

%\tpDef {Christmas}{tenpo [\_sin\_alasa\_noka\_tan\_awen]}{tenpo Santa}{24:00}

%-----------------

\columnbreak

\subsection*{UNITS OF MEASURE}

\tpDef {meter}{linja}{linja}{meter}

\tpDef {kilometer}{linja sewi1 sin}{linja sewi sin}{km, $10^{6}$\,m}

\tpDef {micrometer}{linja sewi1 weka pipi}{linja sewi weka pipi}{$\mu$m, $10^{-6}$\,m}

\tpDef {gram}{anpa}{anpa}{g}

\tpDef {kilogram}{poki sewi1 sin}{anpa sewi sin}{kg, $10^{3}$\,g}

\tpDef {ton}{poki sewi1 suli}{anpa sewi suli}{t, $10^{9}$\,g}

\end{hangparas}
\end{multicols} 


\end{document}
